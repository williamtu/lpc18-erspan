\documentclass[10pt]{sigplanconf}

\usepackage{graphicx}
\usepackage{url}
\usepackage{listings}
\usepackage{amsmath}
\usepackage[font={small}]{caption}
\newcommand{\mycomment}[1]{}
\newcommand{\dfn}[1]{\textit{#1}} % For initial definitions of terms

% For marking text we're probably dropping.
\newcommand{\ignore}[1]{}

% For marking text that we might like to include in a final draft, if
% accepted to RADIO, but that will not fit within the 5-page limit.
\newcommand{\saveforlater}[1]{}

% For marking important work items.
\newcommand{\todo}[1]{\textbf{TODO: #1}}

\lstset{ %
    captionpos=b,
    frame=single
}

\begin{document}

\title{ERSPAN Support for Linux}
\authorinfo{
  \(
  \begin{matrix}
    \textrm{William Tu} & \textrm{Greg Rose} \\
    \textrm{VMware Inc.} & \textrm{VMware Inc.} \\
    \textrm{u9012063@gmail.com} & \textrm{gvrose8192@gmail.com} \\
  \end{matrix}
  \)
}
{}{}

\maketitle

\begin{abstract}
Port mirroring is one of the most common network troubleshooting techiques.
Switch Port Analyzer, SPAN, allows a user to send a copy of the monitored traffic
to a local or remote device using a sniffer or packet analyzer.
Encapsulated Remote SPAN, ERSPAN, extends the basic port mirroring capability
from Layer 2 to Layer 3, allowing the mirrored traffic to be sent through an IP network. 

ERSPAN was added to Linux kernel in 4.14 for IPv4 and 4.16 for IPv6.
In this paper, we demonstrate three ways to use the ERSPAN protocol.
First, using iproute2 to create native tunnel net device.
Traffic sent to the net device will be encapsulated with the protocol header
accordingly and traffic matching the protocol configuration will be received
from the net device.  Second, for eBPF users, using iproute2 to create metadata-mode
ERSPAN tunnel and attach the tunnel metadata implementation in eBPF code.
Finally, Open vSwitch users can use netlink interface to create a switch
and programmatically parse, lookup, and forward the ERSPAN packets based on flows
installed from the userspace.

\end{abstract}

\section{Introduction}\label{introduction}
Port mirroring is one of the most common network troubleshooting techniques.
SPAN (Switch Port Analyzer) allows a user to send a copy of the monitored traffic
to a local or remote device using a sniffer or packet analyzer.
RSPAN is similar, but sends and received traffic on a VLAN. ERSPAN extends the
port mirroring capability from Layer 2 to Layer 3, allowing the mirrored traffic
to be encapsulated in an extension of the GRE (Generic Routing Encapsulation) protocol
and sent through an IP network.  In addition, ERSPAN carries configurable metadatas
(e.g., session ID, timestamps), so that the packet analyzer has better understanding
of the packets.

ERSPAN for IPv4 was added into Linux kernel in 4.14, and for IPv6 in 4.16.
The implementation includes both transmission and reception and is based on the
existing ip\_gre and ip6\_gre kernel module.  As a result, Linux today can act as
an ERSPAN traffic source sending the ERSPAN mirrored traffic to the remote host,
or an ERSPAN destination which receives and parses the ERSPAN packets generated
from Cisco or other ERSPAN-capable switches.
  
We have added both the native tunnel support and metadata-mode tunnel support.
we demonstrate three ways to use the ERSPAN protocol.
First, for Linux users, using iproute2 to create native tunnel net device.
Traffic sent to the net device will be encapsulated with the protocol header
accordingly and traffic matching the protocol configuration will be received
from the net device.  Second, for eBPF users, using iproute2 to create metadata-mode
ERSPAN tunnel.  With eBPF TC hook and eBPF tunnel helper functions, users can
read/write ERSPAN protocol’s fields in finer granularity.
Finally, for Open vSwitch users, using the netlink interface to create a switch
and programmatically parse, lookup, and forward the ERSPAN packets based on flows
installed from the userspace.

\begin{figure}
{\scriptsize
\begin{verbatim}
                             Linux                    Linux
+----------+         +-------------------+      +---------------+
| Physical |         |     VM1  VM2 ...  |      |     linux     |
| Machines |         |      |    |       |      |    netdevs    |
+----------+         |  +--------------+ |      |      |        |
    | links          |  | Open vSwitch | |      | ERSPAN        |
    |                |  +--------------+ |      | netdev        |
+------------+       +-------||----------+      +-||------------+ 
|Cisco switch|               ||                   ||
+------------+               || ERSPAN            || ERSPAN
   || ERSPAN tunnel          || tunnel            || tunnel
   ||                        ||                   ||
    /~~~~~~~~~~~~~~~~~~~~~~~~~~~~~~~~~~~~~~~~~~~~~~\     +-------+ 
    {       Layer 3 IPv4, Ipv6 network             } === |Traffic|
    \~~~~~~~~~~~~~~~~~~~~~~~~~~~~~~~~~~~~~~~~~~~~~~/     |Sniffer|
                                                         +-------+
\end{verbatim}   
}
\vspace{-0.5em}
\caption{Overview of the ERSPAN tunnel use cases. ERSPAN tunnels can be
created from a sniffer machine to Cisco switches, a Linux machine with
multiple VMs, or simply a Linux machine.}
\label{overview}
\vspace{-1.0em}
\end{figure}

ERSPAN is popular in the followin use cases~\cite{erspan_ietf}:
\begin{itemize}
\item Debugging network issues by tracking the control and data frames. 
\item Monitoring Voice-over-IP, VoIP, packets for delay and jitter analysis
\item Monitoring network transactions for latency analysis
\item Monitoring network traffic for anomaly detection
\end{itemize}
Figure~\ref{overview} shows an example setup of ERSPAN tunnels.
A network administrator first sets up multiple source network devices
and filters the interested portion of the traffic he/she wants
to inspect.  One case on the left-most is to create the ERSPAN tunnel
between the Cisco switch and a traffic sniffer.  Depending on the
features in the Cisco switch, different filters can be applied to the
traffic.  In the middle of the figure, for multiple virtual machines running
inside a Linux box, the virtual switch forwarding the packet between
virtual and physical network can also create ERSPAN tunnels between
the software switch and remote traffic sniffer.
Here, Open vSwitch~\cite{ovs} is an example capable of creating filters
and forward packets to ERSPAN tunnels. More detailed configurations of
Open vSwitch are described in later section. 

The ERSPAN tunnel is represented in Linux as a netdev and configured
through iproute2~\cite{iproute2}. Any packet that is placed into its send
queue will be encapsulated based on the netdev's ERSPAN configuration.  
As a result on the right-most, any other linux netdev which wants to create
a ERSPAN mirrored packet simply make a copy and forward to the ERSPAN netdev.
For example, a physical netdev can use linux TC~\cite{tcsubsys} with mirror
action to copy a packet to the erspan tunnel.

Mirrored traffic arrives at the sniffer machine needs to be able to
extract and restore the original monitored frame.  To differentiate
the three use cases, the administrator can create three ERSPAN session
ID, a configuration parameter for grouping the mirrored traffic.
For Linux user, an ERSPAN tunnel can also be used at the sniffer size.
Any packet arriving at the ERSPAN tunnel netdev's receive queue will
be decapsulated. Tools such as Wireshark~\cite{wireshark_erspan,wireshark_erspan2}
can be used to inspect the mirrored packet.

\section{ERSPAN Protocol Implementation}
\begin{figure}
{\scriptsize
\begin{verbatim}
<------------ outer -------------> <---- inner ---- ...
+-+-+-+-+-+-+-+-+-+-+-+-+-+-+-+-+-+-+-+-+-+-+-+-+-+-+-+
| Ether |  IP  |  GRE  |  ERSPAN  | Ether |  IP  | ...   
+-+-+-+-+-+-+-+-+-+-+-+-+-+-+-+-+-+-+-+-+-+-+-+-+-+-+-+
\end{verbatim}
}
\vspace{-1.0em}
\caption{An example of mirrored packet with outer header containing
the GRE and ERSPAN header, followed by the inner Ethernet frame.}
\label{erspanhdr}
\vspace{-1.0em}
\end{figure}

ERSPAN protocol was developed by Cisco and its specification is published at IETF
draft~\cite{erspan_ietf}. 
Figure~\ref{erspanhdr} shows an example of ERSPAN encapsulated
packet, with outer header consists of Ethernet header, followed by IPv4/IPv6
header, followed by a fixed 8-byte GRE header, and followed by ERSPAN header.
After the ERSPAN header, the inner frame is followed so that the ERSPAN receiver
or packet sniffer can extract the original frame.
The use of the IP protocol as part of the outer header is important because it
makes the mirrored traffic routable across any IP network.

ERSPAN protocol has two versions; version 1 (type II) and
version 2 (type III). ERSPAN protocol is layered on top of the GRE (Generic Routing
Encapsulation) protocol, with GRE's sequence number enabled.  For ERSPAN type II,
the GRE's next protocol type is 0x88BE with 8-byte ERSPAN header size, and for
ERSPAN type III, the GRE's next protocol type is 0x22EB with 12-byte ERSPAN header
size, if no optinal subheader enabled.

In this section we describe the basic ERSPAN protocol header format
along with its implementation in the Linux kernel.
For IPv4/IPv6, the implementation is under net/ipv4/ip\_gre.c and
net/ipv6/ip6\_gre.c. Also a userspace API header, include/uapi/linux/erspan.h
is added for metadata-mode tunnel users.

\subsection{Native vs Metadata-Mode Tunnel}
%https://netdevconf.org/1.2/papers/borkmann.pdf
There are two tunnel type implementations in Linux kernel:
native tunnel and metadata-mode tunnel~\cite{lwtunnel}.
Native tunnel is the basic way of creating tunnels in Linux.
A tunnel netdev is created with per tunnel-specific configuration,
tied together with the netdev. For example, creating a GRE tunnel
with key and sequence number can be done by:
\texttt{ip link add dev gre123 type gretap local 1.1.1.1 remote 2.2.2.2 seq key 0xfb}.
As a result, $N$ different tunnel configurations require creating
$N$ number of netdevs.  In certain cases such as network virtualization,
this is not scalable because every host in the network creates
mutiple tunnels with different configurations to every other hosts~\cite{nvp}.

Metada-mode tunnel, or light weight tunnel, is designed for solving
the limitation.  The fundamental idea is that only one netdev per
tunnel type is required to represent multiple tunnels.
This means that the tunnel configuration of a particular type
of the tunnel must be passed to the tunnel netdev in order
to encapsulate the packet.  For example, creating a metadata-mode
tunnel can be done by:
\texttt{ip link add dev type gretap external}.
Note that there is no configuration parameters assigned at device creation
time. The tunnel configuration is set-up per-packet at run-time.
Currently there are two ways of using metadata-mode tunnel, one through
OVS and the other through eBPF~\cite{daniel2}.
We implement both the native mode and metadata-mode~\cite{erspanmd}
for ERSPAN type II and type III.
More examples of using native and metadata-mode tunnel are upstreamed
under tools/testing/selftest/bpf/\{test\_tunnel.sh, test\_tunnel\_kern.c\}.

\subsection{GRE}
ERSPAN follows a fixed 8-byte GRE header with the below value.
{\scriptsize
\begin{verbatim}
GRE header for ERSPAN encapsulation (8 octets [34:41]) -- 8 bytes
0                   1                   2                   3
0 1 2 3 4 5 6 7 8 9 0 1 2 3 4 5 6 7 8 9 0 1 2 3 4 5 6 7 8 9 0 1
+-+-+-+-+-+-+-+-+-+-+-+-+-+-+-+-+-+-+-+-+-+-+-+-+-+-+-+-+-+-+-+-+
|0|0|0|1|0|00000|000000000|00000|    Protocol Type for ERSPAN   |
+-+-+-+-+-+-+-+-+-+-+-+-+-+-+-+-+-+-+-+-+-+-+-+-+-+-+-+-+-+-+-+-+
|      Sequence Number (increments per packet per session)      |
+-+-+-+-+-+-+-+-+-+-+-+-+-+-+-+-+-+-+-+-+-+-+-+-+-+-+-+-+-+-+-+-+
\end{verbatim}
}
Note that only the sequence number bit in the FLAGS fields is set.
Sequence number is useful at the snifffer site where the mirrored
traffic arrives out-of-the-order.  Depending on the protocol type,
ERSPAN type II or type III is followed next.
% note: we do not implement per session seq number?

\textbf{Implementation: }
%commit: gre: add sequence number for collect md mode.
Before introducing ERSPAN, Linux kernel already supports IPv4 GRE native
and metadata mode.  So our effort is to purely add ERSPAN implementation
on top of existing GRE code base~\cite{erspantype2}.
One minor limitation is that existing metadata-mode does not support
GRE sequence number, we've upstreamed the implementation~\cite{greseq}.

For IPv6, there is no metadata mode feature before 4.16.
We first implemented the metadata-mode support for IPv6 GRE~\cite{ip6md},
then upstreamed the ERSPAN feature~\cite{ip6erspanmd, ip6erspan}.
%commit 6712abc168ebac90b46 ip6_gre: add ip6 gre and gretap collect_md mode


\subsection{ERSPAN Type II}
ERSPAN type II has 8-byte feature header with the following format.
{\scriptsize
\begin{verbatim}
ERSPAN Version 1 (Type II) header (8 octets [42:49])
0                   1                   2                   3
0 1 2 3 4 5 6 7 8 9 0 1 2 3 4 5 6 7 8 9 0 1 2 3 4 5 6 7 8 9 0 1
+-+-+-+-+-+-+-+-+-+-+-+-+-+-+-+-+-+-+-+-+-+-+-+-+-+-+-+-+-+-+-+-+
|  Ver  |          VLAN         | COS | En|T|    Session ID     |
+-+-+-+-+-+-+-+-+-+-+-+-+-+-+-+-+-+-+-+-+-+-+-+-+-+-+-+-+-+-+-+-+
|      Reserved         |                  Index                |
+-+-+-+-+-+-+-+-+-+-+-+-+-+-+-+-+-+-+-+-+-+-+-+-+-+-+-+-+-+-+-+-+
\end{verbatim}
}
The ERSPAN Type II encapsulation adds to the original
frame a composite header comprising: 14-byte (802.3) + 20-byte (IP)
+ 8-byte (GRE)  + 8-byte (ERSPAN), in addition to a trailing 4-byte
Ethernet CRC.
The VLAN field shows the original VLAN of the frame, the COS means
Class of Service of the monitored frame.
En field shows the trunk encapsulation type associated with the ERSPAN
source port.
When the mirrored frame is truncated, T bit is set to indicate the
frame has been truncated.
Session ID is a 10-bit field as an identification of each ERSPAN mirring
session.
Index is a platform-depedent field for specifying port number and direction.

%\noindent
\textbf{Implementation: }
Type II introduces two new configurable fields to netlink API;
the Session ID and Index.
Session ID is configured by users through iproute2 tool with netlink API. 
Since ERSPAN does not use the GRE Key field, we re-use the
\texttt{IFLA\_GRE\_IKEY, IFLA\_GRE\_OKEY} as the session ID field.
Index is also configurable by users through iproute2.
The COS field and VLAN field are extrated from the original frame and set
properly.  The truncate bit is detected by comparing the the mirrored frame's
skb->len and the length its IP header reports~\cite{erspantruncate}.
%commit: erspan: auto detect truncated packets.

\subsection{ERSPAN Type III}
ERSPAN type III has 12-byte feature header with the following format.
{\scriptsize
\begin{verbatim}
ERSPAN Version 2 (Type III) header (12 octets [42:49])
0                   1                   2                   3
0 1 2 3 4 5 6 7 8 9 0 1 2 3 4 5 6 7 8 9 0 1 2 3 4 5 6 7 8 9 0 1
+-+-+-+-+-+-+-+-+-+-+-+-+-+-+-+-+-+-+-+-+-+-+-+-+-+-+-+-+-+-+-+-+
|  Ver  |          VLAN         | COS |BSO|T|     Session ID    |
+-+-+-+-+-+-+-+-+-+-+-+-+-+-+-+-+-+-+-+-+-+-+-+-+-+-+-+-+-+-+-+-+
|                          Timestamp                            |
+-+-+-+-+-+-+-+-+-+-+-+-+-+-+-+-+-+-+-+-+-+-+-+-+-+-+-+-+-+-+-+-+
|             SGT               |P|    FT   |   Hw ID   |D|Gra|O|
+-+-+-+-+-+-+-+-+-+-+-+-+-+-+-+-+-+-+-+-+-+-+-+-+-+-+-+-+-+-+-+-+
\end{verbatim}
}
Type III introduces more flexible composite header to support additional
fields.  BSO, Bad/Short/Oversized, allows the sniffer to identify whether
the frame payload has CRC error, too short, or too large~\cite{bso}.
Timestamp is a 4-byte fields and can be configured with different granularities
(100 microseconds, 100 nanosecond, or IEEE1588) at the Gra field.
SGT stands for security group tag of the monitored frame, P field
indicates that the ERSPAN payload is an Ethernet protocol frame.
FT,Frame Type, indicates whether the mirrored frame is a Ethernet
802.3 frame, or a IP packet.
Hardware ID, Hw ID, is an unique identifier of an ERSPAN engine
Direction bit, D, indicates whether the original frame was
SPAN'ed in ingress(0) or in egress(1).
Finally, O indicates whether or not the optinal platform-specific
subheader is presented.

\textbf{Implementation: }
For type III, we introduced another two fields to kernel through
netlink API; hardware ID and direction.
The COS, BSO, and T fields can be extracted or inferred from the mirrored
frame. Timestamp value is calculated by calling the kernel
\texttt{ktime\_get\_real()} with 100 microseconds granularity.
Currently we do not support other timer granulartiy.
In addition, the SGT is hard-coded to 0, non-ethernet mirrored packet
is not supported, so FT is always 0 and P is set to 1.
There is no implementation of sub-headers, so O bit is 0.

\section{Example Use Case}
\begin{figure}
{\scriptsize
\begin{verbatim}
 session 10            Linux, session 20        Linux, session 30
+----------+         +-------------------+      +----------------+
| Physical |         |     VM1  VM2 ...  |      |     linux      |
| Machines |         | 10.1.1.3  |       |      |   netdev,eth1  |
+----------+         |  +--------------+ |      |      |         |
    | links          |  | Open vSwitch | |      | ERSPAN 10.1.1.4|
    | 10.1.1.2       |  +--------------+ |      | netdev         |
+------------+       +-------||----------+      +-||------------+ 
|Cisco switch|               ||                   ||
+------------+               || ERSPAN            || ERSPAN
   || ERSPAN tunnel          || tunnel            || tunnel
   || (192.168.1.2)          || (192.168.1.3)     || (192.168.1.4)
    /~~~~~~~~~~~~~~~~~~~~~~~~~~~~~~~~~~~~~~~~~~~~~~\     +-------+ 
    {       Layer 3 IPv4, Ipv6 network             } === |Traffic|
    \~~~~~~~~~~~~~~~~~~~~~~~~~~~~~~~~~~~~~~~~~~~~~~/     |Sniffer|
                192.168.1.*                              +-------+
                                                        192.168.1.1
\end{verbatim}   
}
\vspace{-0.5em}
\caption{Example of creating three ERSPAN monitoring sessions, $10, 20, 30$,
in a $10.1.1.*$ internal network, mirroring traffic over a $192.168.1.*$ IP network.}
\label{example}
\vspace{-1.0em}
\end{figure}

We use Figure~\ref{example} as an example topology to demostrate
the three configuration ways of ERSPAN.
Assuming a network administrator wants to monitor a network consists
of 1) physical machines connected to Cisco switches, 2)
virtualized Linux machine with multiple VMs deployed and virtual switch
(openvswitch.ko) enforcing the fowarding policies, and
3) non-virtualized Linux physical machines, e.g., service nodes
in data center such as gateways.
Assuming all the monitored servers are under IP network of $10.1.1.*$,
and We place a traffic sniifer over another IP network of $192.168.1.*$.
The following subsection describes the configuration for each case.

\subsection{Cisco Switch}
We use Nexus 5000 switch and configure its ERSPAN tunnel
with session ID 10, remote IP pointing to the traffic sniffer, $192.168.1.1$,
and local IP address as $192.168.1.2$.  As a result, both the ingress and
egress traffic on ports 11 and 12 will be mirrored to the remote sniffer.
\begin{verbatim}
    monitor session 10 type erspan-source
      erspan-id 10
      vrf default
      destination ip 192.168.1.1
      source interface Ethernet1/11 both
      source interface Ethernet1/12 both
      no shut
    monitor erspan origin ip-address 192.168.1.2 global
\end{verbatim}


\subsection{Open vSwitch Kernel Module}\label{ovs}
Open vSwitch consists of two components: a userspace daemon, called ovs-vswitchd,
and a flow cache as a kernel module, called openvswitch.ko.  While ovs-vswitchd
talks to the OpenFlow controller and programs its OpenFlow flow tables,
the openvswitch.ko keeps a cache where the subsequent flows are handled
inside the kernel space.

The openvswitch.ko provides a user-facing netlink API that models a network
bridge that connects multiple ports through a single table~\cite{ovswoovs}.
This example shows how to use the netlink API provided by openvswitch.ko
module, with the utility, ovs-dpctl, to create a ERSPAN tunnel.
\begin{verbatim}
# creating datapath named "mydp", attach veth1(port 2)
ovs-dpctl add-dp mydp
ovs-dpctl add-if mydp veth1 // connected to VM1

# creating erspan dev named "myerspan" and attach
ip link add dev myerspan type erspan external
ovs-dpctl add-if mydp myerspan

# flow entry for port 2 to erspan tunnel 
ovs-dpctl add-flow \
    "in_port(2)" \
    "set(tunnel(tun_id=20,dst=192.168.1.1,ttl=64,\
    erspan(ver=2,dir=1,hwid=0x4),flags(df|key))),3"
\end{verbatim}

%\begin{lstlisting}
%[caption={Example ovs-dpctl usage},language=bash,label=lst:dpctl,breaklines=true]
%\end{lstlisting}

\subsection{iproute2 with/without eBPF}\label{iproute2}
Assuming we want to mirror all traffic from the physical device eth1
to an ERSPAN tunnel with session ID 30, as shown in the right Figure~\ref{example},
we first create a native-mode ERSPAN tunnel using ip-link command,
and mirror traffic from eth1 to the ERSPAN tunnel netdev.
For eBPF use case, instead of creating native-mode tunnel, we create
a metadata-mode tunnel using the key word "external".
Then, tc qdisc and filter rules are created for a eBPF program~\cite{daniel1,daniel2},
"test\_tunnel\_kern.o" with section name "set\_erspan" to be executed,
when receiving a packet from eth1 and sending through the "myerspan"
tunnel device.

%see net/ip6_gre_headroom.sh
\begin{verbatim}
# Native-mode without using eBPF
ip link add dev myerspan type erspan seq key 30 \
    local 192.168.1.4 remote 192.168.1.1 \
    erspan_ver 1 erspan 123
tc filter add dev eth1 ingress pref 1000 matchall \
    skip_hw action mirred egress mirror dev myerspan

# Metadata-mode with eBPF
ip link add dev myerspan type erspan external
tc qdisc add dev myerspan clsact
tc filter add dev myerspan egress bpf direct-action \
    obj test_tunnel_kern.o section erspan_set_tunnel
tc filter add dev eth1 ingress pref 1000 matchall \
    skip_hw action mirred egress mirror dev myerspan
\end{verbatim}

Note that for metadata-mode tunnel, the tunnel configuration is not
provided from the ip route command line, but is passed in to the tunnel
by the eBPF program, test\_tunnel\_kern.o. A code snippet creating this
object from tools/testing/selftests/bpf/test\_tunnel\_kern.c is shown below.
{\scriptsize
\begin{verbatim}
SEC("erspan_set_tunnel")
int _erspan_set_tunnel(struct __sk_buff *skb)
{
        struct bpf_tunnel_key key;
        struct erspan_metadata md; 
        int ret;

        __builtin_memset(&key, 0x0, sizeof(key));
        key.remote_ipv4 = 0xc0a80101; /* 192.168.1.100 */
        key.tunnel_id = 30; // session ID
        key.tunnel_tos = 0;
        key.tunnel_ttl = 64;

        ret = bpf_skb_set_tunnel_key(skb, &key, sizeof(key),
                                     BPF_F_ZERO_CSUM_TX);
        if (ret < 0) {
                ERROR(ret);
                return TC_ACT_SHOT;
        }

        __builtin_memset(&md, 0, sizeof(md));
        md.version = 1;
        md.u.index = bpf_htonl(123);

        ret = bpf_skb_set_tunnel_opt(skb, &md, sizeof(md));
        if (ret < 0) {
                ERROR(ret);
                return TC_ACT_SHOT;
        }

        return TC_ACT_OK;
}
\end{verbatim}
}
\section{Conclusion}
Port mirroring is the most common troubleshooting techniques
allowing a user to send a copy of the monitored traffic to a packet analyzer.
ERSPAN extends its precedence SPAN and RSPAN, by allowing the monitored
traffic to route across IP networks.
In this paper, we describe the implementation of ERSPAN and demonstrate three
ways to use the ERSPAN protocol in Linux, by creating a native-mode ERSPAN tunnel
, or by creating eBPF byte-code with metadata-mode tunnel, or by using Open
vSwitch kernel module. 
We'd like to thank many people for giving review comments feedbacks for our
ERSPAN patches.


\mycomment{
\section{Acknowledgments}
The authors would like to acknowledge Justin Pettit for his guidance on writing
this paper, Martynas Pumputis for providing context around WeaveNet, and Ben
Pfaff, Hannes Sowa and Yingying Ren for providing review feedback.
\section{Author Biography}
Joe Stringer is a software engineer interested in networks and related fields,
in particular SDN. He is actively engaged in the open-source community on
network software projects including Open vSwitch and the Linux networking
stack.
}
\bibliography{paper.bib}
\bibliographystyle{plain}

\end{document}
