\documentclass{sigplanconf}

\usepackage{graphicx}
\usepackage{url}
\usepackage{listings}
\usepackage{amsmath}
\usepackage[font={small}]{caption}

\newcommand{\dfn}[1]{\textit{#1}} % For initial definitions of terms

% For marking text we're probably dropping.
\newcommand{\ignore}[1]{}

% For marking text that we might like to include in a final draft, if
% accepted to RADIO, but that will not fit within the 5-page limit.
\newcommand{\saveforlater}[1]{}

% For marking important work items.
\newcommand{\todo}[1]{\textbf{TODO: #1}}

\lstset{ %
    captionpos=b,
    frame=single
}

\begin{document}

\title{ERSPAN Support for Linux}
\authorinfo{
  \(
  \begin{matrix}
    \textrm{Greg Rose} & \textrm{William Tu} \\
    \textrm{roseg@vmware.com} & \textrm{tuc@vmware.com} \\
  \end{matrix}
  \)
}
{}{}

\maketitle

\begin{abstract}
Port mirroring is one of the most common network troubleshooting techiques.
Switch Port Analyzer, SPAN, allows a user to send a copy of the monitored traffic
to a local or remote device using a sniffer or packet analyzer.
Encapsulated Remote SPAN, ERSPAN, extends the basic port mirroring capability
from Layer 2 to Layer 3, allowing the mirrored traffic to be sent through an IP network. 

ERSPAN was added to Linux kernel in 4.14 for IPv4 and 4.16 for IPv6.
In this paper, we demonstrate three ways to use the ERSPAN protocol.
First, using iproute2 to create native tunnel net device.
Traffic sent to the net device will be encapsulated with the protocol header
accordingly and traffic matching the protocol configuration will be received
from the net device.  Second, for eBPF users, using iproute2 to create metadata-mode
ERSPAN tunnel and attach the tunnel metadata implementation in eBPF code.
Finally, Open vSwitch users can use netlink interface to create a switch
and programmatically parse, lookup, and forward the ERSPAN packets based on flows
installed from the userspace.

\end{abstract}

\section{Introduction}\label{introduction}
Port mirroring is one of the most common network troubleshooting techniques.
SPAN (Switch Port Analyzer) allows a user to send a copy of the monitored traffic
to a local or remote device using a sniffer or packet analyzer.
RSPAN is similar, but sends and received traffic on a VLAN. ERSPAN extends the
port mirroring capability from Layer 2 to Layer 3, allowing the mirrored traffic
to be encapsulated in an extension of the GRE (Generic Routing Encapsulation) protocol
and sent through an IP network.  In addition, ERSPAN carries configurable metadatas
(e.g., session ID, timestamps), so that the packet analyzer has better understanding
of the packets.

ERSPAN for IPv4 was added into Linux kernel in 4.14, and for IPv6 in 4.16.
The implementation includes both transmission and reception and is based on the
existing ip\_gre and ip6\_gre kernel module.  As a result, Linux today can act as
an ERSPAN traffic source sending the ERSPAN mirrored traffic to the remote host,
or an ERSPAN destination which receives and parses the ERSPAN packets generated
from Cisco or other ERSPAN-capable switches.
  
We have added both the native tunnel support and metadata-mode tunnel support.
we demonstrate three ways to use the ERSPAN protocol.
First, for Linux users, using iproute2 to create native tunnel net device.
Traffic sent to the net device will be encapsulated with the protocol header
accordingly and traffic matching the protocol configuration will be received
from the net device.  Second, for eBPF users, using iproute2 to create metadata-mode
ERSPAN tunnel.  With eBPF TC hook and eBPF tunnel helper functions, users can
read/write ERSPAN protocol’s fields in finer granularity.
Finally, for Open vSwitch users, using the netlink interface to create a switch
and programmatically parse, lookup, and forward the ERSPAN packets based on flows
installed from the userspace.

\begin{figure}
{\scriptsize
\begin{verbatim}
                             Linux                    Linux
+----------+         +-------------------+      +---------------+
| Physical |         |     VM1  VM2 ...  |      |     linux     |
| Machines |         |      |    |       |      |    netdevs    |
+----------+         |  +--------------+ |      |      |        |
    | links          |  | Open vSwitch | |      | ERSPAN        |
    |                |  +--------------+ |      | netdev        |
+------------+       +-------||----------+      +-||------------+ 
|Cisco switch|               ||                   ||
+------------+               || ERSPAN            || ERSPAN
   || ERSPAN tunnel          || tunnel            || tunnel
   ||                        ||                   ||
    /~~~~~~~~~~~~~~~~~~~~~~~~~~~~~~~~~~~~~~~~~~~~~~\     +-------+ 
    {       Layer 3 IPv4, Ipv6 network             } === |Traffic|
    \~~~~~~~~~~~~~~~~~~~~~~~~~~~~~~~~~~~~~~~~~~~~~~/     |Sniffer|
                                                         +-------+
\end{verbatim}   
}
\vspace{-0.5em}
\caption{Overview of the ERSPAN tunnel use cases. ERSPAN tunnel can be
created between }
\label{overview}
\vspace{-1.0em}
\end{figure}

ERSPAN is popular in the followin use cases~\ref{erspanietf}:
\begin{itemize}
\item Debugging network issues by tracking the control and data frames. 
\item Monitoring Voice-over-IP, VoIP, packets for delay and jitter analysis
\item Monitoring network transactions for latency analysis
\item Monitoring network traffic for anomaly detection
\end{itemize}
Figure~\ref{fig:overview} shows an example setup of ERSPAN tunnels.
A network administrator first sets up multiple source network devices
and filters the interested portion of the traffic he/she wants
to inspect.  One case on the left-most is to create the ERSPAN tunnel
between the Cisco switch and a traffic sniffer.  Depending on the
features in the Cisco switch, different filters can be applied to the
traffic.  In the middle of the figure, for multiple virtual machines running
inside a Linux box, the virtual switch forwarding the packet between
virtual and physical network can also create ERSPAN tunnels between
the software switch and remote traffic sniffer.
Here, Open vSwitch~\cite{ovs} is an example capabale of creating filters
and forward packets to ERSPAN tunnels. More detailed configurations of
Open vSwitch is described in later section. 

The ERSPAN tunnel is represented in Linux as a netdev and configured
through iproute2~\cite{iproute2}. Any packet that is placed into its send
queue will be encapsulated based on the netdev's ERSPAN configuration.  
As a result on the right-most, any other linux netdev which wants to create
a ERSPAN mirrored packet simply make a copy and forward to the ERSPAN netdev.
For example, a physical netdev can use TC~\cite{tc} to copy a packet
to the erspan tunnel~\cite{tcselftest}.

Mirrored traffic arrives at the sniffer machine needs to be able to
extract and restore the original monitored frame.
For Linux user, an ERSPAN tunnel can also be used at the sniffer size.
Any packet arriving at the ERSPAN tunnel netdev's receive queue will
be decapsulated. Tools such as Wireshark~\cite{wireshark} can be used
to inspect the mirrored packet.

\section{ERSPAN Protocol}
ERSPAN protocol was developed by Cisco and its specification is published at IETF
draft~\cite{erspan}.  
Figure~\ref{fig:erspanhdr} shows an example of ERSPAN encapsulated
packet, with outer header consists of Ethernet header, followed by IPv4/IPv6
header, followed by a fixed 8-byte GRE header, and followed by ERSPAN header.
After the ERSPAN header, the inner frame is followed so that the ERSPAN receiver
or packet sniffer can extract the original frame.
The use of the IP protocol as part of the outer header is important because it
makes the mirrored traffic routable across any IP network.

ERSPAN protocol has two versions; version 1 (type II) and
version 2 (type III). ERSPAN protocol is layered on top of the GRE (Generic Routing
Encapsulation) protocol, with GRE's sequence number enabled.  For ERSPAN type II,
the GRE's next protocol type is 0x88BE with 8-byte ERSPAN header size, and for
ERSPAN type III, the GRE's next protocol type is 0x22EB with 12-byte ERSPAN header
size, if no optinal subheader enabled.
The following subsection describes the two ERSPAN header formats.

\begin{figure}
{\scriptsize
\begin{verbatim}
<------------ outer -------------->  <---- inner ----...
+-+-+-+-+-+-+-+-+-+-+-+-+-+-+-+-+-+-+-+-+-+-+-+-+-+-+-+
| Ether |  IP  |  GRE  |  ERSPAN  | Ether |  IP  | ...   
+-+-+-+-+-+-+-+-+-+-+-+-+-+-+-+-+-+-+-+-+-+-+-+-+-+-+-+
\end{verbatim}
}
\vspace{-1.0em}
\caption{An example of mirrored packet with outer header containing
the GRE and ERSPAN header, followed by the innter Ethernet frame.}
\label{erspanhrd}
\vspace{-1.0em}
\end{figure}

\subsection{GRE header}
{\scriptsize
\begin{verbatim}
GRE header for ERSPAN encapsulation (8 octets [34:41]) -- 8 bytes
0                   1                   2                   3
0 1 2 3 4 5 6 7 8 9 0 1 2 3 4 5 6 7 8 9 0 1 2 3 4 5 6 7 8 9 0 1
+-+-+-+-+-+-+-+-+-+-+-+-+-+-+-+-+-+-+-+-+-+-+-+-+-+-+-+-+-+-+-+-+
|0|0|0|1|0|00000|000000000|00000|    Protocol Type for ERSPAN   |
+-+-+-+-+-+-+-+-+-+-+-+-+-+-+-+-+-+-+-+-+-+-+-+-+-+-+-+-+-+-+-+-+
|      Sequence Number (increments per packet per session)      |
+-+-+-+-+-+-+-+-+-+-+-+-+-+-+-+-+-+-+-+-+-+-+-+-+-+-+-+-+-+-+-+-+
\end{verbatim}
}
Note that in the above GRE header [RFC1701] out of the C, R, K, S,
s, Recur, Flags, Version fields only S (bit 03) is set to 1. The
other fields are set to zero, so only a sequence number follows.

Sequence number is useful in many cases that
when the mirrored traffic is re-ordered.

\subsection{ERSPAN Type II}
ERSPAN type II has 8-byte feature header with the following format.
{\scriptsize
\begin{verbatim}
ERSPAN Version 1 (Type II) header (8 octets [42:49])
0                   1                   2                   3
0 1 2 3 4 5 6 7 8 9 0 1 2 3 4 5 6 7 8 9 0 1 2 3 4 5 6 7 8 9 0 1
+-+-+-+-+-+-+-+-+-+-+-+-+-+-+-+-+-+-+-+-+-+-+-+-+-+-+-+-+-+-+-+-+
|  Ver  |          VLAN         | COS | En|T|    Session ID     |
+-+-+-+-+-+-+-+-+-+-+-+-+-+-+-+-+-+-+-+-+-+-+-+-+-+-+-+-+-+-+-+-+
|      Reserved         |                  Index                |
+-+-+-+-+-+-+-+-+-+-+-+-+-+-+-+-+-+-+-+-+-+-+-+-+-+-+-+-+-+-+-+-+
\end{verbatim}
}

The ERSPAN Type II encapsulation adds to the original
frame a composite header comprising: 14 (802.3) + 20 (IP) + 8 (GRE)
 + 8 (ERSPAN) octets, in addition to a trailing 4-octet Ethernet CRC.
(Note that an 802.1Q encapsulation [802.1Q] would add 4 additional
octets but not reduce the Ethernet MTU size of the container frame.

VLAN riginal VLAN of the frame.

COS
Class of Service of the monitored frame.

En
The trunk encapsulation type associated with the ERSPAN source port for  ingress ERSPAN traffic.
T
Indicates that the frame copy encapsulated in the ERSPAN packet has been truncated
Session ID
Identification associated with each ERSPAN session
Index
Index associated with the ERSPAN traffic's source port and direction, this field is platform dependent.

\subsection{ERSPAN Type III}
{\scriptsize
\begin{verbatim}
ERSPAN Version 2 (Type III) header (12 octets [42:49])
0                   1                   2                   3
0 1 2 3 4 5 6 7 8 9 0 1 2 3 4 5 6 7 8 9 0 1 2 3 4 5 6 7 8 9 0 1
+-+-+-+-+-+-+-+-+-+-+-+-+-+-+-+-+-+-+-+-+-+-+-+-+-+-+-+-+-+-+-+-+
|  Ver  |          VLAN         | COS |BSO|T|     Session ID    |
+-+-+-+-+-+-+-+-+-+-+-+-+-+-+-+-+-+-+-+-+-+-+-+-+-+-+-+-+-+-+-+-+
|                          Timestamp                            |
+-+-+-+-+-+-+-+-+-+-+-+-+-+-+-+-+-+-+-+-+-+-+-+-+-+-+-+-+-+-+-+-+
|             SGT               |P|    FT   |   Hw ID   |D|Gra|O|
+-+-+-+-+-+-+-+-+-+-+-+-+-+-+-+-+-+-+-+-+-+-+-+-+-+-+-+-+-+-+-+-+
\end{verbatim}
}

Type III introduces a larger and more flexible composite header to
support additional fields useful for applications such as network
management, intrusion detection, performance and latency analysis
etc. that require to know all the original parameters of the
mirrored frame, including those not present in the original frame
itself.

The ERSPAN Type III composite header includes a mandatory 12-octet
portion followed by an optional 8-octet platform specific sub-header

ERSPAN Type III supports all of the ERSPAN type II features and
functionality and adds these enhancements:
Provides timestamp information in the ERSPAN Type III header that can
be used to calculate packet latency among edge, aggregate, and core switches.
Identifies possible traffic sources using the ERSPAN Type III header fields

Provides the ability to configure timestamp granularity across all VDCs to
determine how the clock manager synchronizes the ERSPAN timers

\section{Implementation}
The use of the IP protocol as part of the transport is critical to
be able to carry the mirrored traffic across any IP-based network.
The GRE component instead is particularly important to be able to
piggyback different data formats along with the copy of the original
frame.

ERSPAN can carry either  
ERSPAN for IPv4 was implemented under net/ipv4/ip\_gre.c and was
upstreamed to Linux kernel in 4.14~\cite{erspanipv4}.
The ERSPAN for IPv6

\subsection{IPv4}

gre: introduce native tunnel support for ERSPAN

The ERSPAN implementation for IPV
\subsection{IPv6}
\subsection{Limitation}
DF bit is set to prevent fragmentation
• GRE Header protocol type of 0x88BE
• PFC3 and above supports ERSPAN (sup720, sup32)
• Cisco ASR supports ERSPAN as well
• ERSPAN ID uniquely identifies source sessions

%https://sharkfestus.wireshark.org/sharkfest.10/D-2_Chung%20A%20Variety%20of%20Ways%20to%20Capture%20and%20Analyze%20Packets.pdf

\subsection{fragmentation}
\section{Use Cases}
Port mirroring and monitoring overview
Port mirroring is a method of monitoring network traffic that forwards a copy of each incoming or outgoing packet from one port on a network switch to another port where the packet can be analyzed. Port mirroring can be used as a diagnostic tool or debugging feature, especially for preventing attacks. Port mirroring can be managed locally or remotely.
You can configure port mirroring, by assigning a port (known as the Monitor port), from which the packets are copied and sent to a destination port (known as the Mirror port). All packets received on the Monitor port or issued from it, are forwarded to the second port. You next attach a protocol analyzer on the mirror port to monitor each segment separately. The analyzer captures and evaluates the data without affecting the client on the original port.

\subsection{Linux iproute2}\label{iproute2}
\subsection{eBPF}\label{ebpf}
\subsection{Open vSwitch}\label{ovs}
\subsection{Cisco Switch}
    gre: introduce native tunnel support for ERSPAN

\begin{verbatim}
    monitor session 100 type erspan-source
      erspan-id 123
      vrf default
      destination ip 172.16.1.200
      source interface Ethernet1/11 both
      source interface Ethernet1/12 both
      no shut
    monitor erspan origin ip-address 172.16.1.100 global
\end{verbatim}

\section{Conclusion}
\section{Acknowledgments}
\section{Future Work}
\end{document}
