\documentclass{sigplanconf}

\usepackage{graphicx}
\usepackage{url}
\usepackage{listings}
\usepackage{amsmath}
\usepackage[font={small}]{caption}

\newcommand{\dfn}[1]{\textit{#1}} % For initial definitions of terms

% For marking text we're probably dropping.
\newcommand{\ignore}[1]{}

% For marking text that we might like to include in a final draft, if
% accepted to RADIO, but that will not fit within the 5-page limit.
\newcommand{\saveforlater}[1]{}

% For marking important work items.
\newcommand{\todo}[1]{\textbf{TODO: #1}}

\lstset{ %
    captionpos=b,
    frame=single
}

\begin{document}

\title{ERSPAN Support for Linux}
\authorinfo{
  \(
  \begin{matrix}
    \textrm{Greg Rose} & \textrm{William Tu} \\
    \textrm{roseg@vmware.com} & \textrm{tuc@vmware.com} \\
  \end{matrix}
  \)
}
{}{}

\maketitle

\begin{abstract}
Port mirroring is one of the most common network troubleshooting techniques.
SPAN (Switch Port Analyzer) allows a user to send a copy of the monitored traffic
to a local or remote device using a sniffer or packet analyzer.
RSPAN is similar, but sends and received traffic on a VLAN. ERSPAN extends the
port mirroring capability from Layer 2 to Layer 3, allowing the mirrored traffic
to be encapsulated in an extension of the GRE (Generic Routing Encapsulation) protocol
and sent through an IP network.  In addition, ERSPAN carries configurable metadatas
(e.g., session ID, timestamps), so that the packet analyzer has better understanding
of the packets.

ERSPAN for IPv4 was added into Linux kernel in 4.14, and for IPv6 in 4.16.
The implementation includes both transmission and reception and is based on the
existing ip\_gre and ip6\_gre kernel module.  As a result, Linux today can act as
an ERSPAN traffic source sending the ERSPAN mirrored traffic to the remote host,
or an ERSPAN destination which receives and parses the ERSPAN packets generated
from Cisco or other ERSPAN-capable switches.
  
We’ve added both the native tunnel support and metadata-mode tunnel support.
In this paper, we demonstrate three ways to use the ERSPAN protocol.
First, for Linux users, using iproute2 to create native tunnel net device.
Traffic sent to the net device will be encapsulated with the protocol header
accordingly and traffic matching the protocol configuration will be received
from the net device.  Second, for eBPF users, using iproute2 to create metadata-mode
ERSPAN tunnel.  With eBPF TC hook and eBPF tunnel helper functions, users can
read/write ERSPAN protocol’s fields in finer granularity.
Finally, for Open vSwitch users, using the netlink interface to create a switch
and programmatically parse, lookup, and forward the ERSPAN packets based on flows
installed from the userspace.

\end{abstract}

\section{Introduction}\label{introduction}

\end{document}
